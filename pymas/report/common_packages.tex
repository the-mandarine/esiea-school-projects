%--- PREREQUIS --- %
\usepackage{ifthen} % Options Si-Sinon pour les commandes latex

%Est-ce qu'on est sur un beamer ?
\newif\ifbeamer
\ifthenelse{\isundefined{\beamerbutton}}{\beamerfalse}{\beamertrue}

%--- GENERAL ---%
\usepackage{ucs}
\usepackage[utf8x]{inputenc} % Encodage du fichier
\usepackage[T1]{fontenc}  % Encodage des polices du document
\usepackage[english,francais]{babel} % Langages utilisés
\usepackage{lmodern} % Police pour rendre le texte sélectionable
\usepackage{pifont} % For access to weird symbols
\newcommand{\carriagereturn}{\Pisymbol{psy}{191}}

%\usepackage{xstring}

\ifbeamer
\else
\usepackage[top=2.5cm, bottom=2.5cm, left=2.5cm, right=2.5cm]{geometry} % Marges PDF
%\usepackage[top=2.5cm, bottom=2.5cm, left=2.5cm, right=3.5cm]{geometry} % Marges impression livre
\fi
\setlength{\parskip}{1ex plus .4ex minus .4ex}


%---BEAMER --- %
\ifbeamer
\usetheme{Darmstadt}
\useoutertheme[footline=institutetitle,subsection=true]{miniframes}
\setbeamertemplate{frametitle}[default][center] %titre centré
\addtobeamertemplate{footline}{\hfill\insertframenumber/\inserttotalframenumber\hspace{2em}\null}
%\usecolortheme{orchid}
%\usecolortheme{crane}

%Pour la transparence des pdf
\pdfpageattr {/Group << /S /Transparency /I true /CS /DeviceRGB>>} 
\fi

%--- STRUCTURES DONNEES ---%
\usepackage{enumerate} % Listes
\usepackage{array} % Tableaux
\usepackage{tabularx} % Tableaux modulables
\ifbeamer
\else
\usepackage[table]{xcolor} % Tableaux couleur
\fi
\usepackage{graphicx} % Figures
\usepackage[normalsize,up,bf,center]{caption2} % Meilleur distinction des figures
\renewcommand{\captionfont}{\bfseries \itshape}
\renewcommand{\captionlabeldelim}{~--}

% -- TABLEAUX ---- %
\usepackage{longtable} % Tableaux sur plusieurs pages
\usepackage{multicol} % Colonnes
\usepackage{multirow} % Lignes
\usepackage{rotating} % Écriture et figure verticales
\renewcommand{\arraystretch}{1.5} % Meilleur lisibilité des tableaux.

%--- MISE EN FORME & CARACTERES SPECIAUX ---%
\usepackage{amssymb} % Symboles
\usepackage{eurosym} % € symbol
\usepackage[autolanguage]{numprint} % 1000 => 1 000
\ifbeamer
\else
\usepackage[colorlinks=true,urlcolor=blue,pdfstartview=FitH,linkcolor=blue]{hyperref} % Liens
\fi
\usepackage{ulem} % Souligner, barrer, etc.
\usepackage{relsize} % Taille relative \smaller \bigger
\usepackage{color} % Couleur
\definecolor{darkred}{rgb}{0.72,0.16,0.11}
\definecolor{darkgreen}{rgb}{0.00,0.60,0.25}
\definecolor{darkblue}{rgb}{0.00,0.00,0.70}
\definecolor{brightblue}{rgb}{0.00,0.50,0.95}
\definecolor{darkpurple}{rgb}{0.58,0.07,0.49}
\definecolor{palegray}{gray}{0.80}
\definecolor{lightgray}{gray}{0.70}
\definecolor{pinkcomment}{rgb}{1,0,1}
\definecolor{keywordred}{rgb}{0.65,0.17,0.23}
\definecolor{pink}{rgb}{1,0.39,1}
\definecolor{violet}{rgb}{0.41,0.35,0.8}

\newcommand{\showcarriagereturn}{\color{green}\carriagereturn}
\newcommand{\showspace}{\color{green}\textvisiblespace}


%--- CODES SOURCES & CO ---%
\usepackage{verbatim} % Texte brut
\usepackage[french]{algorithm2e} % Algorithmes
\newenvironment{algo}[0]{\vrule~\vrule\itshape \begin{algorithm}[H] }{\normalfont\end{algorithm}}
\newcommand{\var}[1]{\textnormal{\texttt{#1}}}
\RestyleAlgo{plain}
\SetKwSwitch{Selon}{Cas}{Autre}{selon}{faire}{cas où}{autres cas}{fin}
\SetKwRepeat{Repeter}{r\'ep\'eter}{tant que}
\SetKwFor{uTq}{tant que}{faire}{} % Tant que sans fin (équivaut à uSi)
\SetKwFor{oTq}{}{}{fin} % Tant que déjà ouvert, donc sans balise de départ
\SetKwIF{oSi}{}{}{}{}{}{}{fin}
\SetKwIF{oFn}{}{}{fonction}{}{}{}{fin}
\SetKwFunction{KwFn}{Fn}



\usepackage{listings} % Codes sources
\usepackage{textcomp} % nécessaire pour upquote
\lstloadlanguages{[ANSI]C, sh, Python} % Listes de langages autorisés

% [style = sourceC] for the C codes
\lstdefinestyle{sourceC}{language=C, basicstyle=\small\color{black}\ttfamily, classoffset=0, keywordstyle=\color{blue}\bfseries, classoffset=1, morekeywords={NULL}, keywordstyle=\color{magenta}\bfseries, commentstyle=\color{brightblue}\itshape, stringstyle=\color{darkgreen}, showstringspaces=false, tabsize=3, framexleftmargin=2mm, frame=shadowbox, rulesepcolor=\color{lightgray}, breaklines=true, emph={main, printf, scanf, FILE, fopen, fscanf, fprintf, fclose, rand}, emphstyle=\color{darkblue}\bfseries,columns=fullflexible, flexiblecolumns=true, upquote=true, keepspaces=true}
% spaceflexible 

 
% Le langage C par défaut et le ANSI. On ne précise pas [ANSI] devant car cela pose problème si utilise le style dans un tableau 

% [style = python] for the python codes
\lstdefinestyle{python}{language=python, basicstyle=\small\color{black}\ttfamily, classoffset=0, keywordstyle=\color{blue}\bfseries, classoffset=1, morekeywords={NULL}, keywordstyle=\color{magenta}\bfseries, commentstyle=\color{brightblue}\itshape, stringstyle=\color{darkgreen}, showstringspaces=false, tabsize=3, framexleftmargin=2mm, frame=shadowbox, rulesepcolor=\color{lightgray}, breaklines=true, emph={main, printf, scanf, FILE, fopen, fscanf, fprintf, fclose, rand}, emphstyle=\color{darkblue}\bfseries,columns=fullflexible, flexiblecolumns=true, upquote=true, keepspaces=true}

% [style = msgTerminal] for shell codes
\lstdefinestyle{msgTerminal}{language=sh, basicstyle=\footnotesize\color{white}\ttfamily, keywordstyle=\color{white}, commentstyle=\color{white}\itshape, stringstyle=\color{white}, showstringspaces=false, framexleftmargin=3mm, xleftmargin=3mm, frame=none, tabsize=3, backgroundcolor=\color{darkgray}, rulecolor=\color{black}, breaklines=true,columns=fullflexible,upquote=true}

\lstdefinestyle{msgTerminalW}{language=sh, basicstyle=\small\color{black}\ttfamily, keywordstyle=\color{black}, commentstyle=\color{black}\itshape, stringstyle=\color{black}, showstringspaces=false, framexleftmargin=3mm, xleftmargin=3mm, frame=none, tabsize=3, backgroundcolor=\color{white}, rulecolor=\color{white}, breaklines=true,columns=fullflexible,upquote=true}

% [style = bash] for bash scripts
\lstdefinestyle{bash}{language=sh, basicstyle=\small\color{black}\ttfamily, classoffset=0, keywordstyle=\color{keywordred}\bfseries, morekeywords={mkdir, touch}, classoffset=1, morekeywords={$,$monNom,$RANDOM,$0,$1,$2,$3,$\#,$min,$max,$nomRep,$nom,$val,$elem}, keywordstyle=\color{magenta}\bfseries, commentstyle=\color{darkblue}\bfseries, stringstyle=\color{darkgreen}, showstringspaces=false, tabsize=3, framexleftmargin=2mm, frame=shadowbox, rulesepcolor=\color{lightgray}, breaklines=true, emph={main, printf, scanf, FILE, fopen, fscanf, fprintf, fclose}, emphstyle=\color{darkblue}\bfseries,columns=fullflexible,upquote=true}

%keywordstyle=\color{white}, commentstyle=\color{white}\itshape, stringstyle=\color{white}, showstringspaces=false, framexleftmargin=3mm, xleftmargin=3mm, frame=none, tabsize=3, backgroundcolor=\color{black}, rulecolor=\color{black}, breaklines=true,columns=fullflexible}

% [style = inlineSourceC] for C expression into text
\lstdefinestyle{inlineSourceC}{language=C, basicstyle=\footnotesize\color{black}\ttfamily, classoffset=0, keywordstyle=\color{blue}\bfseries, classoffset=1, morekeywords={NULL}, keywordstyle=\color{magenta}\bfseries, commentstyle=\color{brightblue}\itshape, stringstyle=\color{darkgreen}, showstringspaces=false, tabsize=3, breaklines=true, emph={main, printf, scanf, FILE, fopen, fscanf, fprintf, fclose, rand}, emphstyle=\color{darkblue}\bfseries,columns=fullflexible, keepspaces=true,upquote=true}
% Le langage C par défaut et le ANSI. On ne précise pas [ANSI] devant car cela pose problème si utilise le style dans un tableau (ne pas préciser frame=none non plus pour la même raison)

% [style = inlineTerminal] for shell expression into text
\lstdefinestyle{inlineTerminal}{language=sh, basicstyle=\footnotesize\color{black}\ttfamily\bfseries, keywordstyle=\color{black}, commentstyle=\color{black}\itshape, stringstyle=\color{black}\itshape, showstringspaces=false, tabsize=3, breaklines=true,upquote=true}
% Ne pas préciser frame=none pour laisser la possibilité d'utiliser le style dans un tableau

% [style = cadre] for printing an empty answer box (use framextopmargin=XXmm to set the height)
\lstdefinestyle{cadre}{language=[ANSI]C, basicstyle=\small\color{red}\ttfamily, showstringspaces=false, tabsize=3, framexleftmargin=2mm, frame=single, framerule=0.2mm, rulecolor=\color{black}, breaklines=true, columns=fullflexible, showlines=true,upquote=true}

% [style = sourceJava] for the Java codes
\lstdefinestyle{sourceJava}{language=Java, basicstyle=\small\color{black}\ttfamily, classoffset=0, keywordstyle=\color{darkgreen}\bfseries, commentstyle=\color{brightblue}\itshape, stringstyle=\color{darkgreen}, showstringspaces=false, tabsize=3, framexleftmargin=2mm, frame=shadowbox, rulesepcolor=\color{lightgray}, breaklines=true, emph={main, printf, FILE, fopen, fscanf, fprintf, fclose,new,return,this,super,null,break,continue,if,else,switch,case,default,do,while,for,break,new,return}, emphstyle=\color{darkred}\bfseries,columns=fullflexible,upquote=true}
% Le langage Java par défaut

% [style = sourceJava] for the Java codes
\lstdefinestyle{inlineSourceJava}{style=sourceJava, basicstyle=\normalsize\color{black}\ttfamily}
% Le langage Java par défaut


%--- CHEMINS DES DOSSIERS IMAGES---%
\graphicspath{{../../Commons/Images/} {./img/} {../../../../Commons/Images/}{../../../../../Commons/Images/}{../../../../../../Commons/Images/}{../../../Commons/Images/}}

% TODO : réintroduire une option pour se passer du paragraphe à la demande (ancienement none)

%FOR EACH pour parser différent arguments séparé par une virgule
%SPLIT pour séparer key=value
%Source : http://stackoverflow.com/questions/2402354/split-comma-separated-parameters-in-latex
%Plus bidouille A. Gademer (pour en déduire split)

\makeatletter

% Functional foreach construct 
% #1 - Function to call on each comma-separated item in #3
% #2 - Parameter to pass to function in #1 as first parameter
% #3 - Comma-separated list of items to pass as second parameter to function #1
\def\foreach#1#2#3{%
  \@test@foreach{#1}{#2}#3,\@end@token%
}

% Internal helper function - Eats one input
\def\@swallow#1{}

% Internal helper function - Checks the next character after #1 and #2 and 
% continues loop iteration if \@end@token is not found 
\def\@test@foreach#1#2{%
  \@ifnextchar\@end@token%
    {\@swallow}%
    {\@foreach{#1}{#2}}%
}

% Internal helper function - Calls #1{#2}{#3} and recurses
% The magic of splitting the third parameter occurs in the pattern matching of the \def
\def\@foreach#1#2#3,#4\@end@token{%
  #1{#2}{#3}%
  \@test@foreach{#1}{#2}#4\@end@token%
}

% Example-function used in foreach, which takes two params and builds hrefs
%\def\makehref#1#2{\href{#1/#2}{#2}}

% Using foreach by passing #1=function, #2=constant parameter, #3=comma-separated list
%\foreach{\makehref}{http://stackoverflow.com}{2409851,2408268}

% Will in effect do
%\href{http://stackoverflow.com/2409851}{2409851}\href{http://stackoverflow.com/2408268}{2408268}



% Functional Split construct 
% #1 - Function to call on each key=value items in #2
% #2 - key=value items to pass as parameter to function #1
\def\split#1#2{%
  \@test@split{#1}#2\@end@token%
}

% Internal helper function - Checks the next character after #1 and 
% continues loop iteration if \@end@token is not found 
\def\@test@split#1{%
  \@ifnextchar\@end@token%
    {\@swallow}%
    {\@split{#1}}%
}



% Internal helper function - Calls #1{#2}{#3}
% The magic of splitting the second parameter occurs in the pattern matching of the \def
\def\@split#1#2=#3\@end@token{%
  #1{#2}{#3}
}

\makeatother

% Example-function used in split, which takes two params
%\def\showbiz#1#2{Je vois #1 et #2.}

% Using foreach by passing #1=function, #2= key=value items
%\foreach{\showbiz}{key=value}

% Will in effect do
%Je vois key et value


% FIN FOR EACH et SPLIT




% PUPUCES !!!!! %

\newcommand{\pupuceC}[4]{\begin{center}\includegraphics[width=0.9cm]{#1} \color{#2}\\\textbf{#3}~\\\ifthenelse{\equal{#1}{quote}}{\itshape \og~#4~\fg}{#4}\end{center}}
\newcommand{\pupuceCN}[3]{\begin{center}\includegraphics[width=0.9cm]{#1} \color{#2}\\\textbf{#3}\vspace{-0.5cm}\end{center}}
 \newcommand{\pupuceLN}[4]{\includegraphics[height=0.6cm]{#1} \begin{minipage}{0.9\textwidth}  \color{#2}\textbf{#3}\end{minipage}}
 \newcommand{\pupuceL}[4]{\includegraphics[height=0.6cm]{#1} \begin{minipage}{0.9\textwidth}  \color{#2}\textbf{#3}~\ifthenelse{\equal{#1}{quote}}{\itshape \og~#4~\fg}{#4}\end{minipage}}
\newcommand{\pupuce}[5]{\ifthenelse{\equal{#1}{C}}{\pupuceC{#2}{#3}{#4}{#5}}{\ifthenelse{\equal{#1}{L}}{\pupuceL{#2}{#3}{#4}{#5}}{\ifthenelse{\equal{#1}{CN}}{\pupuceCN{#2}{#3}{#4}}{\ifthenelse{\equal{#1}{LN}}{\pupuceLN{#2}{#3}{#4}}{}}}}}

%--- DEFINITION DES ANNOTATIONS ---%

\newcommand{\error}[3][C]{\pupuce{#1}{error}{darkred}{#2}{#3}} % appel \error{TITRE}{TEXTE}

\newcommand{\warning}[3][C]{\pupuce{#1}{warning}{darkred}{#2}{#3}} % appel \warning{TITRE}{TEXTE}

\newcommand{\help}[3][C]{\pupuce{#1}{help}{brightblue}{#2}{#3}} % appel \help{TITRE}{TEXTE}

\newcommand{\info}[3][C]{\pupuce{#1}{information}{brightblue}{#2}{#3}} % appel \info{TITRE}{TEXTE}

\newcommand{\save}[3][C]{\pupuce{#1}{save}{darkgreen}{#2}{#3}} % appel \save{TITRE}{TEXTE}

\newcommand{\tips}[3][C]{\pupuce{#1}{tick}{darkgreen}{#2}{#3}} % appel \tips{TITRE}{TEXTE}

\newcommand{\ubuntu}[3][C]{\pupuce{#1}{UbuntuCoF}{orange}{#2}{#3}} % appel \ubuntu{TITRE}{TEXTE}

\newcommand{\quoted}[3][C]{\pupuce{#1}{quote}{darkpurple}{#2}{#3}} % appel \ubuntu{TITRE}{TEXTE}

\newcommand{\panicadvise}[3][C]{\pupuce{#1}{dont_panic}{brightblue}{#2}{#3}}

%\quoted{TITRE}{TEXTE}

\newcommand{\eureka}[0]{\includegraphics[height=36pt]{eureka}}
\newcommand{\dontpanic}[0]{\includegraphics[height=24pt]{dont_panic}}
\newcommand{\emosmile}[0]{\includegraphics[height=12pt]{emosmile}}




% CITATION DES SOURCES DES IMAGES
\newcommand{\bysa}[1]{\scriptsize{#1~\includegraphics[height=12pt]{cc_cc_30}\includegraphics[height=12pt]{cc_by_30}\includegraphics[height=12pt]{cc_sa_30}}}
\newcommand{\source}[1]{\begin{footnotesize} Source~:~#1 \end{footnotesize}}
\newcommand{\vsource}[1]{\begin{footnotesize} \color{gray} ~\begin{sideways}  \mbox{#1}\end{sideways} \end{footnotesize} }
\newcommand{\wikimedia}[1]{Wikimédia/#1}


%\begin{tableOfCommands}{nbcol}{disposition}{header}{légende}
% données du tableau
%\end{tableOfCommands}
\newenvironment{tableOfCommands}[4]{\begin{longtable}{#2}
\caption{#4}\tabularnewline
\hline   #3 \tabularnewline
\hline
\hline
\endfirsthead
\hline  #3  \tabularnewline
\hline
\hline 
\endhead
\multicolumn{#1}{c}{ (Suite page suivante)}\\
\endfoot
\hline
\endlastfoot}{\end{longtable}\medskip}

\newenvironment{tableOfCommandsNoCapt}[3]{\begin{longtable}{#2}
\hline   #3 \tabularnewline
\hline
\hline
\endfirsthead
\hline  #3  \tabularnewline
\hline
\hline 
\endhead
\multicolumn{#1}{c}{ (Suite page suivante)} \tabularnewline
\hline 
\endfoot
\hline
\endlastfoot}{\end{longtable}\medskip}

% % % % % % % % % % % % % % % % % % % % % % % % % % % % % % % % % % %
% Affichage de la division euclidienne et méthode de Horner !
% % % % % % % % % % % % % % % % % % % % % % % % % % % % % % % % % % %

\makeatletter
\newbox\nb@box
\newcount\nb@a
\newcount\nb@b
\newcount\iter@
\newcommand\division[2][2]{%
   \def\dividende@{#2}\def\base@{#1}\iter@\@ne\division@{#2}{#1}}
\newcommand\division@[2]{%
   \setbox\nb@box\hbox{\kern0.5em#1\kern0.5em}%
   \nb@a#1 \nb@b#1 \divide\nb@b#2
   \vtop{%
      \begingroup
         \multiply\nb@b#2 \advance\nb@a-\nb@b
         \hbox to\wd\nb@box{\hfil#1\hfil}%
         \vskip3pt\hrule height0pt width\wd\nb@box\vskip3.4pt
         \hbox to\wd\nb@box{\hfil\textbf{\color{red}\number\nb@a}\kern0.5em}%
         \expandafter\xdef\csname reste@\number\iter@\endcsname{\number\nb@a}%
      \endgroup}%
   \setbox\nb@box\hbox{8}\vrule height\ht\nb@box depth3.5ex
   \setbox\nb@box\hbox{\kern0.5em\ifnum#2>\nb@b #2\else\number\nb@b\fi\kern0.5em}%
   \vtop{%
      \hbox to\wd\nb@box{\kern0.5em#2\hfil}%
      \vskip3pt\hrule height0.4pt width\wd\nb@box\vskip3pt
      \hbox{%
         \csname @\ifnum\nb@b>\z@ first\else second\fi oftwo\endcsname
         {\advance\iter@\@ne\gdef\maxiter{\number\iter@}%
          \expandafter\division@\expandafter{\number\nb@b}{#2}}%
         {\kern0.5em\number\nb@b\xdef\maxiter{\number\iter@}}}}}

\newcommand\afficheresultat{$(\dividende@)_{10}=(\afficheresultat@\maxiter)_{\base@}$}
\newcommand\afficheresultat@[1]{%
   \csname reste@#1\endcsname
   \ifnum#1>\@ne
      \expandafter\expandafter\expandafter\afficheresultat@
   \else
      \expandafter\@gobble
   \fi{\number\numexpr#1-1}}
\makeatother

% % % % % % % % % % % % % % % % % % % % % % % % % % % % % % % % % % %
% Fin Affichage de la division euclidienne et méthode de Horner !
% % % % % % % % % % % % % % % % % % % % % % % % % % % % % % % % % % %



% Balises pour les questions & les exercices %
%\theoremstyle{definitionTP}
\usepackage[nocut]{thmbox}
\thmboxoptions{bodystyle=\noindent}
\newtheorem[L]{questionTheo}{\textbf{QUESTION}}
\newtheorem[L]{exerciceTheo}{\textbf{EXERCICE}}
\newtheorem[L]{correctionTheo}{\textbf{CORRECTION}}


\newenvironment{definitionTP}[1]{
\begin{tabular}{|m{1.3cm}m{0.8\textwidth}|}
\hline
\vspace{0.2cm}\includegraphics[width=1.3cm]{definition}&
\bfseries\underline{#1}~:}{
\tabularnewline
\hline
\end{tabular}
}




% L'environnement question est une encapsulation du théroème questionTheo
% On l'appelle de manière simple par \begin{question} ... \end{question}
% Si l'on veut marquer un type de question (Bonus, Défis), on peut écrire :
% \begin{question}[name=Défis} ... \end{question]
% ATTENTION, none est obligatoire si on a le paramètre optionnel
% Si l'on veut décrire le type de question, on peut écrire :
% \begin{question}[name=Défis,footnote=blah blah} ... \end{question]
% blah blah apparaitra en note de bas de page. :-)
%
% Attention à bien mettre la deuxième paire d'accollades (même vide) !
% Sinon vous perdez la première lettre de votre phrase
%
\makeatletter

%\newcommand{\setThmCut}[1][false]{\ifthenelse{\equal{#1}{false}}{\thmbox@cutfalse}{\thmbox@cuttrue}}

\newcommand{\parseExoArgs}[2]{
%Je vois #1 et #2.
\ifthenelse{\equal{#1}{cut}}{\ifthenelse{\equal{#2}{true}}{\thmbox@cuttrue}{}}{}% Découpage autorisé ?
\ifthenelse{\equal{#1}{name}}{\def\@name{#2}}{}%
\ifthenelse{\equal{#1}{label}}{\def\@label{#2}}{}%
\ifthenelse{\equal{#1}{footnote}}{%
\def\@footnote{\footnotemark}
\def\@footnoteTxt{#2}
}{}% Découpage autorisé ?
}

\newenvironment{exoreponse}[3]{
%A mettre au début de l'environnement.
\medskip%
\def\@name{none}%
\def\@footnote{}%
\def\@footnoteTxt{none}%
\def\@label{none}
\def\@theo{#2}%
\foreach{\split}{\parseExoArgs}{#1}% Parsing des options
\ifthenelse{\equal{\@name}{none}}{% Pas de noms (entre parenthèse)
\begin{\@theo}% 
}{%sinon
\begin{\@theo}[\@name\@footnote]%
}%
\ifthenelse{\not\equal{\@label}{none}}{%
\label{\@label}%
}{}%
\ifthenelse{\not\equal{#3}{none}}{% 
\includegraphics[height=20pt]{#3}%On affiche l'image associée
}{}%
}{% A mettre à la fin (attention, pas accès à #1 ni #2)
\end{\@theo}%
\ifthenelse{\not\equal{\@footnoteTxt}{none}}{\footnotetext{\@footnoteTxt}}{}%
}% Fin definition newenvirronment
\makeatother

\newenvironment{exercice}[1][]{
\begin{exoreponse}{#1}{exerciceTheo}{doit}%
}{%
\end{exoreponse}%
}% Fin definition newenvirronment

\newenvironment{question}[1][]{
\begin{exoreponse}{#1}{questionTheo}{question}%
}{%
\end{exoreponse}%
}% Fin definition newenvirronment

\newenvironment{correctionExo}[1][]{
\begin{exoreponse}{#1}{correctionTheo}{none}%
\color{red}\bfseries%
}{%
\end{exoreponse}%
}% Fin definition newenvirronment


%--- OBJECTIF TP EN AVANT PROPOS ---%
\newcommand{\abstractTP}[1]{\section*{Avant propos} \textit{#1}} % appel \abstractTP{TEXTE}

\newcommand{\difficulty}[1]{~\includegraphics[height=24pt]{difficult_#1}\protect\footnote{La difficulté estimé concerne le niveau et le temps de \underline{réflexion} nécessaire, pas la complexité algorithmique ou de programmation, qui sont à votre portée.}}



